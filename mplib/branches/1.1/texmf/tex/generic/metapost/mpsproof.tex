% This is a modification of mproof.tex (as found on CTAN in
% /systems/knuth/local/lib). The name was changed to mpsproof.tex
% to honor the license restrictions.
%
%%%% These are the comments of the original mproof.tex:
%% To produce proof sheets from MetaPost output (includes date and time):
%%   tex mproof <MetaPost output file names>
%%%%
%
% In contrast, mpsproof allows either pdftex or ordinary tex to be used:
%
%    [pdf]tex mpsproof [\noheaders] <mpost output file names>
%
% Examples:     pdftex  mpsproof  figur.1  figur.2
%               tex  mpsproof  \noheaders  figur.1  figur.2  figur.3
%
% With \noheaders, the filenames, datestamp and page numbers are omitted.
% Using mpsproof one can also produce encapsulated output:
%
%    [pdf]tex mpsproof \encapsulate <one_mpost_output_file_name>
%
% Examples:     pdftex  mpsproof  \encapsulate  figur.1
%               tex  mpsproof  \encapsulate  figur.3
%
% "Encapsulated" means that the pdf output (or the .ps resulting from
% dvips mpsproof.dvi), has the same "bounding box" as the MetaPost figure.
% Actually, it produces a file whose *page size* is the same as that
% bounding box.
%
% If pdftex is used and \pdfoutput = 1, the result will be a pdf file,
% otherwise the result is a dvi file containing dvips graphics \specials.
%
% Limitations: the production of pdf requires supp-pdf.tex and the
% macros therein use up TeX's memory. The specific portion that is used
% is the "save_size". A lot of short paths is no problem, but a single
% long path might exceed the limits on your system. On most systems the
% limit can be increased by editing texmf.cnf.
%
% Placed in the public domain by Daniel H. Luecking.
%
% Change HISTORY:
%    2003/04/08 -- Original release
%    2004/05/02 -- Allow "_" in filenames.
%    2004/05/07 -- Added the \noheaders option.
%    2005/10/07 -- sanitize the filename (\meaning)
%    2006/04/06 -- encapsulation added
%    2006/06/30 -- Better documentation (the above comments)
%
% Test for pdf:
\newif\ifpdf
\pdffalse
\ifx\pdfoutput\undefined
  \else\ifx\pdfoutput\relax
  \else\ifnum\pdfoutput<1
  \else \pdftrue
\fi\fi\fi

% Graphic inclusion macros:
\ifpdf
  \input supp-pdf
  \def\includemps#1{\convertMPtoPDF{#1}{1}{1}}
  \def\MPSpresets{\pdfoutput=1 \pdfcompresslevel=9
    \chardef\makeMPintoPDFobject=1\relax}
  \def\encapspecial{\pdfpageheight=\ht0 \pdfpagewidth=\wd0}
\else
  \input epsf
  \def\includemps#1{\hbox{\epsfverbosetrue\epsfbox{#1}}}
  \def\MPSpresets{}
  \def\encapspecial{\special{papersize=\the\wd0,\the\ht0}}
\fi

\let\EA\expandafter
\def\EAEAEA{\EA\EA\EA}

\def\processMPfile{%
  \MPSpresets \hsize=100in \vsize=\hsize
  \hoffset=-1in \voffset=\hoffset \topskip=0pt
  \setbox0=\vbox{\EA\includemps\EA{\theMPfile}}%
  \ifdim\wd0<1in \message{[warning:  width<1in]}\fi
  \ifdim\ht0<1in \message{[warning: height<1in]}\fi
  \encapspecial \box0 \end}

% To omit printing of files name, page numbers and date stamp:
\newif\ifheaders\headerstrue
\def\noheaders{\headersfalse}

% produce output with same bounding box:
\newif\ifencapsulate
\def\encapsulate{\noheaders\encapsulatetrue}

\def\maybeprint#1{\ifheaders#1\fi}
\def\stripprefix#1>{}

\font\textt=cmtex10 \textt
\def\okbreak{\vfil\penalty2000\vfilneg}

\def\today{\ifcase\month\or
  January\or February\or March\or April\or May\or June\or
  July\or August\or September\or October\or November\or December\fi
  \space\number\day, \number\year}
\newcount\m \newcount\n
\n=\time \divide\n 60 \m=-\n \multiply\m 60 \advance\m \time
\def\hours{\twodigits\n\twodigits\m}
\def\twodigits#1{\ifnum #1<10 0\fi \number#1}

\footline={\ifheaders\sevenrm\today\ at \hours\hss\tenrm\folio\hss\fi}

% DHL - \convertMPtoPDF redefines \\, so we replace it with \?.
\def\endit{\?}

\def\munch#1 {\def\theMPfile{#1}%
  \ifx\theMPfile\endit
    \EA\endit
  \else
    % sanitize the filename:
    \edef\theMPfile{\EA\stripprefix\meaning\theMPfile}%
    \ifencapsulate \EAEAEA\processMPfile
    \else \leavevmode\maybeprint\theMPfile
      $$\EA\includemps\EA{\theMPfile}$$\par\okbreak%
    \fi
  \fi\munch}

\def\?{\let\munch=\relax\def\?{}}

\everypar{\everypar{}\relax\munchline}

\parindent 0pt
% We need the space before \? to ensure that it doesn't end up
% in a filename.
\obeylines\def\munchline#1
{\catcode`\^^M=5 \munch#1 \? \vfill\end}%
