% Meant to be processed with isol1-t1.tcx.
% 
%       dec   oct   hex ISO Latin 1	Cork
% ^@      0     0  0x00 .notdef		grave
% ^L     12   014  0x0c .notdef		ogonek
% M-^^  158  0236  0x9e ogonek		dbar
% M-^_  159  0237  0x9f circumflex	section
% M-'   167  0247  0xa7 section		gbreve
% M-8   184  0270  0xb8 ogonek		ydieresis      
% M-_   223  0337  0xdf	ss		SS
% M-p   240  0360  0xf0 eth		eth
% M-^?  255  0377  0xff	ydieresis	ss

\catcode0 = 12 % so TeX doesn't just ignore it!

% Input is: null (not printable, appears as the three chars `^^@'),
%           ^^df (prints as ^^ff)
%           ^^f0 (prints as itself)
%           ^^ff (prints as ^^b8).
\def\mystring{���}

% ``Prints'' means in the DVI file -- using a Cork-encoded font, of course.
\font\cork = pplr8t \cork  \mystring\ (should be: grave ss eth ydieresis).

% in \message, the character undergoes translation back to the external
% code, so it effectively appears as itself.
\message{three chars ^ ^ @
   ss-0337/df)
  eth-0360/f0)
  ydieresis-0377/ff: \mystring}

%\message{^^ff ^^df �}
%get:   � � �
%should?� ^^df �
%\message{\number`�}
%get:   223
%should? 255

\tracingoutput = 1
\bye
