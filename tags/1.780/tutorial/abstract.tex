\begin{abstract}
  Individuals that use \TeX{} (or any of its derivatives) to typeset
  their documents generally take extra measures to ensure paramount
  visual quality.  Such documents often contain mathematical expressions
  and graphics to accompany the text.  Since \TeX{} was designed ``for
  the creation of beautiful books\Dash and especially for books that
  contain a lot of mathematics''~\cite{knuth:texbook}, it is clear that
  it is sufficient (and in fact \textit{exceptional}) at dealing with
  mathematics and text.  \TeX{} was not designed for creating graphics;
  however, certain add-on packages can be used to create modest figures.
  \TeX{}, however, is capable of including graphics created with other
  utilities in a variety of formats.  Because of their scalability,
  Encapsulated PostScript (\EPS) graphics are the most common types
  used.  This paper introduces \MP{} and demonstrates the fundamentals
  needed to generate high-quality \EPS{} graphics for inclusion into
  \TeX-based documents.
\end{abstract}
