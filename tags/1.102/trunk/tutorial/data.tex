\begin{section}{Data types}
  There are ten data types in \MP{}: \textit{numeric}, \textit{pair},
  \textit{path}, \textit{transform}, \textit{rgbcolor},
  \textit{cmykcolor}, \textit{string}, \textit{boolean},
  \textit{picture}, and \textit{pen}.  These data types allow users to
  store fragments of the graphics for later use.  We will briefly
  discuss each of these data types and elaborate on how they are used in
  a typical \MP{} program.
  \renewcommand{\labelitemi}{$\diamond$}
\begin{itemize}
	\item \textit{numeric}\Dash numbers
	\item \textit{pair}\Dash ordered pairs of numerics
	\item \textit{path}\Dash B\'{e}zier curves (and lines)
	\item \textit{picture}\Dash pictures
	\item \textit{transform}\Dash transformations such as shifts,\\rotations, and slants
	\item \textit{rgbcolor} or \textit{color}\Dash triplets with each component between $0$ and
    $1$ (red, green, and blue)
	\item \textit{cmykcolor}\Dash quadruplets with each component between $0$ and
    $1$ (cyan, magenta, yellow, and black)
	\item \textit{string}\Dash strings of characters
	\item \textit{boolean}\Dash ``true'' or ``false'' values
	\item \textit{pen}\Dash stroke properties
\end{itemize}

Virtually all programming languages provide a way of storing and
retrieving numerical values.  This is precisely the purpose of the
\textit{numeric} data type in \MP.  Since graphics drawn with \MP{} are
simply two dimensional pictures, it is clear that an ordered pair is
needed to identify each point in the picture.  The \textit{pair} data
type provides this functionality.  Each point in the plane consists of
an $x$ (i.e., abscissa) part and a $y$ (i.e., ordinate) part.  \MP{}
uses the standard syntax for defining points in the plane, e.g.,
\texttt{(}\textit{x}\texttt{,}\textit{y}\texttt{)} where both $x$ and
$y$ are numeric data typed variables.

In order to store paths between points, the \textit{path} data type is
used.  All paths in \MP{} are represented as cubic B\'{e}zier curves.
Cubic B\'{e}zier curves are simply parametric splines of the form
$(x(t),y(t))$ where both $x(t)$ and $y(t)$ are piecewise cubic
polynomials of a common parameter $t$.  Since B\'{e}zier curves are
splines, they pairwise interpolate the points.  Furthermore, cubic
B\'{e}zier curves are diverse enough to provide a ``smooth'' path
between all of the points for which it interpolates.  \MP{} provides
several methods for affecting the B\'{e}zier curve between a list of
points.  For example, piecewise linear paths (i.e., linear splines) can
be drawn between a list of points since all linear polynomials are also
cubic polynomials.  Furthermore, if a specific direction for the path is
desired at a given point, this constraint can be forced on the
B\'{e}zier curve.

The \textit{picture} data type is used to store an entire picture for
later use.  For example, in order to create animations, usually there
are objects that remain the same throughout each frame of the animation.
So that these objects do not have to be manually drawn for each frame, a
convenient method for redrawing them is to store them into a picture
variable for later use.

When constructing pairs, paths, or pictures in \MP{}, it is often
convenient to apply affine transformations to these objects.  As
mentioned above, Figure \ref{fig:circles} can be constructed by rotating
the same circle several times before drawing it.  \MP{} provides
built-in affine transformations as ``building blocks'' from which other
transformations can be constructed.  These include shifts, rotations,
horizontal and vertical scalings, and slantings.

For creating colored graphics, \MP{} provides two data types:
\textit{rgbcolor} and \textit{cmykcolor}.  These data types correspond
to the two supported color models \RGB{} and \CMYK.  While using the
\RGB{} color model, fractions of the primary colors
\textit{red}~\showcol{red}, \textit{green}~\showcol{green}, and
\textit{blue}~\showcol{blue} are ``additively mixed''.  Similarly, in
the \CMYK{} color model, the primary colors
\textit{cyan}~\showcol{cyan}, \textit{magenta}~\showcol{magenta},
\textit{yellow}~\showcol{yellow}, and \textit{black}~\showcol{black} are
``subtractively mixed.''  The former model is suitable for on-screen
viewing whereas the latter model is preferred in high-quality print.
Both color types are ordered tuples, $(c_1,c_2,c_3)$ and
$(c_1,c_2,c_3,c_4)$, with components~$c_i$ being \textit{numeric}s
between $0$ and $1$.  For example, in the \RGB{} color model, a light
orange tone can be referred to as
\texttt{(1,.6,0)}~\showcol[rgb]{1,.6,0}, whereas in the \CMYK{} color
model \texttt{(0,.6,1,0)}~\showcol[cmyk]{0,.6,1,0} corresponds to a
clearly different orange tone.  If a particular color is to be used
several times throughout a figure, it is natural to store this color
into a variable of type \textit{rgbcolor} or \textit{cmykcolor}.

The data type \textit{color} is a convenient synonym for
\textit{rgbcolor}.  Additionally, there are five built-in \RGB{} colors
in \MP{}: \texttt{black}, \texttt{white}, \texttt{red}, \texttt{green},
and \texttt{blue}.  So, the expression \texttt{.4(red+blue)} refers to a
dark violet~\showcol[rgb]{.4,0,.4} in the \RGB{} color model and in the
example above \texttt{(1,.6,0)} could be replaced by
\texttt{red+.6green}.

The most common application of \textit{string} data types is reusing a
particular string that is typeset (or labeled).  The \textit{boolean}
data type is the same as in other programming languages and is primarily used in
conditional statements for testing.  Finally, the \textit{pen} data type
is used to affect the actual stroke paths.  The default unit of
measurement in \MP{} is $1\,\mathrm{bp}=1/72\mathrm{\ in}$, and the
default thickness of all stroked paths is $0.5\,\mathrm{bp}$.  An
example for using the \textit{pen} data type may include changing the
thickness of several stroked paths.  This new pen can be stored and then
referenced for drawing each of the paths.
\end{section}
