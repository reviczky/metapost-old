\subsection{The \texttt{draw} command}

The most common command in \MP{} is the \texttt{draw} command.  This
command is used to draw paths or pictures.  In order to draw a path from
\texttt{z1:=(0,0)} to \texttt{z2:=(54,18)} to \texttt{z3:=(72,72)}, we
should first decide how we want the path to look.  For example, if we
want these points to simply be connected by line segments, then we
use

\begin{center}
  \verb|draw z1--z2--z3;|
\end{center}

However, if we want a smooth path between these points, we use

\begin{center}
  \verb|draw z1..z2..z3;|
\end{center}

In order to specify the direction of the path at the points, we use the
\texttt{dir} operator.  In Figure \ref{fig:draw1} we see that the smooth
path is horizontal at \texttt{z1}, a 45\textdegree\ angle at
\texttt{z2}, and vertical at \texttt{z3}.  These constraints on the
B\'{e}zier curve are imposed by

\begin{center}
  \verb|draw z1{right}..z2{dir 45}..{up}z3;|
\end{center}

\begin{figure}[hptb]
	\begin{center}
    \textattachfile[color={0 0 0},mimetype={text/plain}]{draw.mp}{\includegraphics{draw-1.mps}}
  \end{center}
	\caption{\texttt{draw} examples}
  \label{fig:draw1}
\end{figure}

Notice that \verb|z2{dir 45}| forces the \textit{outgoing} direction at
\texttt{z2} to be 45\textdegree.  This implies an \textit{incoming}
direction at \texttt{z2} of 45\textdegree.  In order to require
different incoming and outgoing directions, we would use

\begin{center}
  \verb|draw z1{right}..{dir |$\theta_i$\verb|}z2{dir |$\theta_o$\verb|}..{up}z3;|
\end{center}

where $\theta_i$ and $\theta_o$ are the incoming and outgoing
directions, respectively.
